\documentclass[margin,line]{res}

\oddsidemargin -.5in
\evensidemargin -.5in
\textwidth=6.0in
\itemsep=0in
\parsep=0in
\addtolength{\textheight}{.6in}
% if using pdflatex:
%\setlength{\pdfpagewidth}{\paperwidth}
%\setlength{\pdfpageheight}{\paperheight} 

\newenvironment{list1}{
  \begin{list}{\ding{113}}{%
      \setlength{\itemsep}{0in}
      \setlength{\parsep}{0in} \setlength{\parskip}{0in}
      \setlength{\topsep}{0in} \setlength{\partopsep}{0in} 
      \setlength{\leftmargin}{0.17in}}}{\end{list}}
\newenvironment{list2}{
  \begin{list}{$\bullet$}{%
      \setlength{\itemsep}{0in}
      \setlength{\parsep}{0in} \setlength{\parskip}{0in}
      \setlength{\topsep}{0in} \setlength{\partopsep}{0in} 
      \setlength{\leftmargin}{0.2in}}}{\end{list}}


\begin{document}
\name{Benjamin Tannenwald \hspace{50mm}Curriculum Vitae\vspace*{.1in}}

\begin{resume}
\section{\sc Contact Information}
\vspace{.05in}
\begin{tabular}{@{}p{2in}}
Physics Research Building, 3043 \\
191 West Woodruff Avenue \\
Columbus, Ohio, USA \\
+1 (614) 648 4323 \\
benjamin.tannenwald@cern.ch \\
\end{tabular}


\section{\sc Education}
{\bf Ph.D.} Ohio State University, Columbus, Ohio, USA\hfill August 2017\\
\vspace*{-.22in}
%\begin{list1}
%\item[] Ph.D., Experimental Particle Physics, expected Winter 2016
\begin{list2}
\vspace*{.1in}
\item Dissertation Title:  ``Measurement of the helicity fractions of $W$ bosons in semileptonic $t\bar{t}$ decays and a search for exotic di-higgs production in the $b\bar{b}WW^*$ final state with the ATLAS detector at the LHC''
\item Advisor:  Harris Kagan
\end{list2}
%\end{list1}
%\vspace*{.1in}

{\bf M.Sc.} Ohio State University, Columbus, Ohio, USA\hfill August 2013\\
\vspace*{-.2in}

{\bf B.S., B.A.} University of Kansas, Lawrence, Kansas, USA \hfill May 2011\\
%\vspace*{-.1in}
%\begin{list1}
%\item[] B.A., Anthropology, Mathematics, May 2011
%\item[] B.S., Physics, May 2011
%\end{list1}

\section{\sc Positions}
Postdoctoral Researcher, Ohio State University \hfill August 2017 - Present \\
Graduate Research Associate, Ohio State University/CERN \hfill August 2013 - August 2017 \\
Graduate Teaching Associate, Ohio State University \hfill September 2011 - May 2013

\section{\sc Societies}
American Physical Society

\section{\sc Honors/Awards}
{2015 CERN-Fermilab HCP Summer School, CERN, July 2015}\\
{Hazel Brown Outstanding Teaching Assistant Award, Ohio State University, Spring 2012}\\
{Stranathan Scholarship for Outstanding Senior of Physics, University of Kansas, June 2010} \\

\section{\sc Research Interests}
LHC physics, Higgs decays, top quark physics, di-higgs searches, $t\bar{t}$ measurements, statistical interpretations of data, hardware and software development
\vspace*{.1in}
\section{\sc Research Summary} 
{\bf Measuring $W$-helicity fractions at 8 TeV in semileptonic top quark pair decays}\\
I was one of two analyzers on the 8 TeV measurement of the $W$ boson helicity fractions in semileptonic $t\bar{t}$ decays. My contributions included the analysis of the hadronic $W$ decays, the kinematic fitting procedure used to reconstruct the $t\bar{t}$ system, and the template fitting procedure used to extract the helicity fractions from the final observables. This work produced the most sensitive $W$ helicity fraction result from ATLAS and was the first analysis to directly measure helicity angles using the hadronic $W$ in $t\bar{t}$ decays. The analysis has been been published in the European Physical Journal C (EPJC). 

{\bf Search for exotic di-higgs production at 13 TeV using the $b\bar{b}WW^*$ final state}\\
I am one of five analyzers working on a Run 2 search for exotic di-higgs production in the semileptonic $b\bar{b}WW^*$ final state. This is the first search ever attempted in this channel. Using the knowledge and techniques gained from my experience with $t\bar{t}$, I am responsible for developing the data-driven methods to estimate multi-jet background contributions, using kinematic fitting techniques to reduce contributions coming from $t\bar{t}$, and developing and implementing the statistical analysis used to set final limits on exotic di-higgs production. The analysis is nearing completion and is aiming to have public results for by Autumn 2017.

\newpage

{\bf Measuring vector boson scattering in the $V\gamma$ final state}\\
I worked as one of the analyzers on the 8 TeV measurement of vector boson scattering in the $V\gamma$ final state. I contributed to the intial sensitivity studies, optimization of the event selection, and a data-driven estimation of the $V$+jets contributions in the various signal and control regions. The analysis has been published in the Journal of High Energy Physics (JHEP).

%\newpage

{\bf Measurement of the Higgs boson decay into $Z\gamma$}\\
I contributed to the Run I analysis searching for Higgs boson decays in the $Z\gamma$ channel. I worked on estimating and modelling the inteference between Higgs and other Standard Model production in the $\ell\ell\gamma$ final state.

{\bf DCS Interface for the ATLAS Diamond Beam Monitor }\\ 
I developed the software interface for control and monitoring of the ATLAS Diamond Beam Monitor (DBM) using the WinCC language platform. I also aided in the installation of the DBM and tests of the detector's proper configuration and operation.

{\bf Trigger and shift leader shifts in the ATLAS Control Room} \\ 
I have taken both trigger and shift leader shifts as part of the Run II operation of the ATLAS detector.
\vspace{5pt}


\section{\sc Recent Publications}
``Search for Higgs boson pair production in the $b\bar{b}WW^*$ final state at $\sqrt{s}=13$ TeV the ATLAS detector'', with M. Aaboud et. al. [ATLAS Collaboration], In preparation\vspace{8pt} \\
``Measurement of the $W$ Boson Helicity Fractions in $t\bar{t}$ Events at $\sqrt{s}=8$ TeV in the Lepton + Jets Channel with ATLAS'', with M. Aaboud et. al. [ATLAS Collaboration], Eur. Phys. J. C 77 (2017) 264\vspace{8pt} \\
``Measurements of $Z\gamma$ electroweak production in association with a high-mass dijet system at $\sqrt{s}=$ 8 TeV with the ATLAS detector'', with M. Aaboud et. al. [ATLAS Collaboration], JHEP, Vol. 07, p. 107, 2017\vspace{8pt} \\
``Search for Higgs boson decays to a photon and a $Z$ boson in $pp$ collisions at $\sqrt{s}=$ 7 and 8 TeV with the ATLAS detector'', with M. Aaboud et. al. [ATLAS Collaboration], Phys. Lett. B, 732, 8 (2014)\vspace{8pt} \\
%\vspace*{.05in}
%\vspace*{.7in}
\vspace{-15pt}

\section{\sc Recent Talks\\ And Presentations}
``Search for Higgs boson pair production in the $hh\rightarrow b\bar{b}WW^*\rightarrow b\bar{b}\ell\nu q\bar{q}$ final state using the ATLAS detector'', DPF 2017, Fermilab, Illinois, USA, July 2017 \vspace{8pt}\\
``Measurement of the $W$ boson helicity fractions in $t\bar{t}$ events in the lepton+jets channel using the ATLAS detector at $\sqrt{s}=$ 8 TeV", APS April Meeting, Salt Lake City, Utah, USA, April 2016 \vspace{8pt}\\
``Differential cross-section measurements of top quark pair production at 8 TeV with the ATLAS detector", TOP2015, Ischia, Italy, September 2015 \vspace{8pt}\\
``Vector boson plus jets measurements with the ATLAS detector", QCD@LHC, London, UK, September 2015 \vspace{8pt}\\
``Measuring W-helicity using Semileptonic $t\bar{t}$ Decays at 8 TeV", ATLAS Week, CERN, June 2015 \\
%\vspace*{.1in}
%\vspace*{.7in}
\vspace{-10pt}

\section{\sc Skills}
{\bf Computer}: ROOT, C++, Python, Linux, Shell Scripting, \LaTeX, Grid Computing, WinCC \\
{\bf Languages}: English (native), French (fluent)

%\section{\sc Education/\\Outreach}
%{\bf ATLAS Tour Guide} \hfill December 2013 - September 2015\\
%I directed and led tours of the ATLAS control room and experimental cavern for groups visiting CERN, introducing the basic principles behind the physics done at CERN and explaining how ATLAS is able to detect and measure the particles produced from collisions.
\vspace{10pt} 
\section{\sc References}
Available upon request
%\newpage

%\section{\sc References}
%Prof. Harris Kagan \\
%Department of Physics  \\
%Ohio State University \\
%Columbus, OH 43210 \\
%U.S.A.\\
%Email: kagan.1@osu.edu   \\                                                            
%Phone: (614) 292-7331  \\

%Prof. Biagio Di Micco \\
%Dipartimento di Matematica e Fisica \\
%Universita Roma Tre \\
%Rome, 00146 \\
%Italy \\
%Email: biagio.di.micco@cern.ch\\ 
%Phone: +39 65 733 7265 \\

%Dr. Elizaveta Shabalina \\
%Department of Physics, Institute II\\
%Georg-August-Universit{\"a}t G{\"o}ttingen \\
%G{\"o}ttingen, D-37077 \\
%Germany\\
%Email: elizaveta.shabalina@phys.uni-goettingen.de\\ 
%Phone: +41 22 76 77627\\

%Prof. K.K. Gan \\
%Department of Physics \\
%Ohio State University \\
%Columbus, OH 43210\\
%U.S.A. \\
%Email: gan.1@osu.edu\\ 
%Phone: (614) 292-4124\\

\end{resume}
\end{document}
