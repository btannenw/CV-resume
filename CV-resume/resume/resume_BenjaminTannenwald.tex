\documentclass[line]{letter}
%\documentclass[margin,line]{letter}
%\oddsidemargin -.2in
%\evensidemargin -.7in
%\textwidth=6.5in
%\addtolength{\textheight}{.6in}

\usepackage{enumitem}
\usepackage[textwidth=7in,textheight=9in]{geometry}

\newenvironment{list1}{
  \begin{list}{\ding{113}}{%
      \setlength{\itemsep}{0in}
      \setlength{\parsep}{0in} \setlength{\parskip}{0in}
      \setlength{\topsep}{0in} \setlength{\partopsep}{0in} 
      \setlength{\leftmargin}{0.17in}}}{\end{list}}
\newenvironment{list2}{
  \begin{list}{$\bullet$}{%
      \setlength{\itemsep}{0in}
      \setlength{\parsep}{0in} \setlength{\parskip}{0in}
      \setlength{\topsep}{0in} \setlength{\partopsep}{0in} 
      \setlength{\leftmargin}{0.2in}}}{\end{list}}



\pagenumbering{gobble}
\begin{document}

{\bf \Large Benjamin Tannenwald \large \hspace{20mm}btannenwald@gmail.com \hfill 614-648-4323\vspace*{.05in}}
%\noindent\makebox[10cm]{\rule{\paperwidth}{2pt}}
\\\noindent{\rule{\textwidth}{1.5pt}}

%\begin{resume}
\vspace{.1in}
{\Large \bf Summary\vspace{5pt}\\}
Particle physicist transitioning to private-sector data science. Extensive experience with analysis of large data
sets using C++ and Python. Excellent critical thinking, problem solving, and communication skills.

{\vspace{10pt}\Large \bf Education\vspace{7pt}\\}
{\bf  Ph.D.} in experimental particle physics, Ohio State University, Columbus, OH\hfill 2011 - 2017\\
%{\bf M.Sc.} Ohio State University, Columbus, Ohio, USA\hfill 2011 - 2013\\
{\bf B.S.} in physics, University of Kansas, Lawrence, KS\hfill 2007 - 2011\\
{\bf B.A.} in mathematics and anthropology,  University of Kansas, Lawrence, KS\hfill 2007 - 2011
%\begin{list1}
%\item[] B.A., Anthropology, Mathematics, May 2011
%\item[] B.S., Physics, May 2011
%\end{list1}

{\vspace{10pt}\Large \bf Work Experience\vspace{7pt}\\}
{\bf Graduate Research Associate}, Ohio State University/CERN \hfill 2013 - 2017 \\
\vspace{-3mm}
\begin{list1}
\item[] Worked in large data analysis collaboration to perform precise measurements of fundamental particles and search for new signatures of the Higgs boson with petabyte scale datasets collected at the CERN Large Hadron Collider
\begin{itemize}
\item Developed C++ software frameworks used by physicists at Ohio State University, Indiana University, Oxford University (UK), Georg-August-Universit{\"a}t G{\"o}ttingen (Germany), and Universita Roma Tre (Italy)
\item Designed algorithms to sift through terabytes of physics data, categorize events based on significant criteria, and optimize reconstruction of subatomic interactions using combinatorical likelihoods 
\item Created data-driven methods for modeling background processes in searches for novel experimental signatures 
\item Experience with statistical techniques used to model complex multivariate processes and calculate significance of observed results
\end{itemize}\vspace{2mm}
\item[] Worked extensively with the control software of the ATLAS detector, designing and commissioning user interfaces
\begin{itemize}
\item Tested and installed sensitive silicon and diamond-based pixel detectors
\item Used WinCC to create data structures and interface for efficient detector monitoring and operation
\end{itemize}\vspace{2mm}
\item[] Presented analysis work and results to research groups inside the collaboration and to larger audiences at international scientific conferences\vspace{2mm}
\item[] Developed and maintained Condor computing cluster used for simulation and data analysis
\end{list1}
\vspace{2mm}
{\bf Graduate Teaching Associate}, Ohio State University \hfill 2011 - 2013 \\
\vspace{-6mm}
\begin{itemize}
\item Taught undergraduate holography, general physics classes both with and without calculus, and lab courses (each 20-30 students)
\item Received Hazel Brown Outstanding Teaching Assistant Award in 2012
\end{itemize}


%{\bf \sc Research Interests}
%LHC physics, Higgs decays, top quark physics, di-higgs searches, $t\bar{t}$ measurements, statistical interpretations of data, hardware and software development
%\vspace*{.1in}

%{\bf \sc Research Summary} 
%{\bf Measuring $W$-helicity fractions at 8 TeV in semileptonic top quark pair decays}\\
%I was one of two analyzers on the 8 TeV measurement of the $W$ boson helicity fractions in semileptonic $t\bar{t}$ decays. My contributions included the analysis of the hadronic $W$ decays, the kinematic fitting procedure used to reconstruct the $t\bar{t}$ system, and the template fitting procedure used to extract the helicity fractions from the final observables. This work produced the most sensitive $W$ helicity fraction result from ATLAS and was the first analysis to directly measure helicity angles using the hadronic $W$ in $t\bar{t}$ decays. The analysis has been been published in the European Physical Journal C (EPJC). 

%{\bf Search for exotic di-higgs production at 13 TeV using the $b\bar{b}WW^*$ final state}\\
%I am one of five analyzers working on a Run 2 search for exotic di-higgs production in the semileptonic $b\bar{b}WW^*$ final state. This is the first search ever attempted in this channel. Using the knowledge and techniques gained from my experience with $t\bar{t}$, I am responsible for developing the data-driven methods to estimate multi-jet background contributions, using kinematic fitting techniques to reduce contributions coming from $t\bar{t}$, and developing and implementing the statistical analysis used to set final limits on exotic di-higgs production. The analysis is nearing completion and is aiming to have public results for by Summer 2017.

%\newpage

%{\bf Measuring vector boson scattering in the $V\gamma$ final state}\\
%I worked as one of the analyzers on the 8 TeV measurement of vector boson scattering in the $V\gamma$ final state. I contributed to the intial sensitivity studies, optimization of the event selection, and a data-driven estimation of the $V$+jets contributions in the various signal and control regions. The analysis has been published in the Journal of High Energy Physics (JHEP).

%{\bf Measurement of the Higgs boson decay into $Z\gamma$}\\
%I contributed to the Run I analysis searching for Higgs boson decays in the $Z\gamma$ channel. I worked on estimating and modelling the inteference between Higgs and other Standard Model production in the $\ell\ell\gamma$ final state.

%{\bf DCS Interface for the ATLAS Diamond Beam Monitor }\\ 
%I developed the software interface for control and monitoring of the ATLAS Diamond Beam Monitor (DBM) using the WinCC language platform. I also aided in the installation of the DBM and tests of the detector's proper configuration and operation.

%{\bf Trigger and shift leader shifts in the ATLAS Control Room} \\ 
%I have taken both trigger and shift leader shifts as part of the Run II operation of the ATLAS detector.

{\vspace{10pt}\Large \bf Skills\vspace{-12pt}\\}
\begin{itemize}[leftmargin=*]
\item Extensive experience with data analysis in C++ involving large volume data sets on distributed computing
systems with Condor and distributed computing%\vspace{-2mm}

\item Comfortable with Python, Git, shell scripting, \LaTeX, WinCC, and Mathematica%\vspace{-2mm}

\item Strong oral and written communication skills. Experience giving both formal and informal presentations%\vspace{-2mm}

\item Works well in teams, both in leadership and supporting roles

\item Accustomed to collaborating with both large and
small groups across time zones and continents%\vspace{-2mm}

\item Fluent in French
\end{itemize}

%\end{resume}
\end{document}
